\chapter{Conclusion}
After developing the app and writing the dissertation, there were definitely some parts of the project that were easier than others. Firstly, research played a major factor in what framework, platforms and language was used to make a functional application for users and servers. We outlined how the project would be done throughout this document and what we needed to have achieved as objectives which can be seen in the Introduction. A Flask server was set up which then accessed the MongoDB.
Using Google Maps, users’ locations were shown once locations services were working. A orange and white colour scheme was added to give a fun and complimentary looking app. The most common colour scheme is of course blue, as it is a sign of peace and order; A login class was created with activities working correctly. This would then ask for a name and password or if not signed up, the option to do so would be implemented. As progression with the app was made, meetings with our supervisors would happen and feedback between both collaborators would be regular. The Meetings we felt were successful and helpful in some stages when developing the app. The background section outlined the technologies used and why we came to choose each of them. This clears up how different technologies did or did not suit our project. Testing during the development stage is seen as it gives an overview of the steps involved. 
Android development was a good choice in the end when it came to developing the application. There was not many issues using this technology as it is written in Java, a programming language which is universally known and has plenty of resources when it comes to development. The mobile application was easy to deploy on phones for testing which was a feature that was enjoyable. MongoDB and Flask were relatively new technologies used for this project. This was difficult as there were not many ways of finding information on any issues that emerged. Regarding the meetings, they were successful, and the application was changed if there were any issues and completed in the way that it was intended for the user. This kind of Application has not been completed very successfully before, in this way, from researching and testing other spot finder applications.
Time Management was done efficiently in the later stages of the project, whereas towards the beginning, there was a little bit of wasted time on the back-end. But, as we already had a plan on how this project would look, the research and development portion were not taking up as much time as anticipated.
